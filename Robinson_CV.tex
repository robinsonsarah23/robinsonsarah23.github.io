%DOCUMENT CLASS
\documentclass[11pt]{article}

%PACKAGES
\usepackage{amsmath}
\usepackage[shortlabels]{enumitem} %Allows more enumerate options
\usepackage{fancyhdr}
\usepackage[left=.5in,right=.5in,top=.5in,bottom=0.6in,footskip=0.2cm, headheight=0cm, headsep=0cm]{geometry} %Sets page margins
\usepackage{hyperref}
\usepackage{lastpage}
\usepackage{ltablex}
\usepackage{tabularx}
\usepackage[super]{nth}
\usepackage{datetime}
\usepackage{setspace}
\usepackage{booktabs}
\usepackage{ulem}
\usepackage{comment}

%HEADER FOOTER
\pagestyle{fancy}
\renewcommand{\headrulewidth}{0pt}
\cfoot{\textit{\thepage\ of \pageref*{LastPage}}}
\newdateformat{monthyeardate}{\monthname[\THEMONTH] \THEYEAR}
\lfoot{\textit{Sarah Robinson}}
\rfoot{\textit{Updated \monthyeardate\today}}

% adjustments for first page
\fancypagestyle{firstpage}{
\lfoot{}
}

%COLUMN WIDTHS
\keepXColumns
\newcolumntype{Y}{>{\hsize=.36\hsize}X}
\newcolumntype{Z}{>{\hsize=1.64\hsize}X}

%OTHER CUSTOM SETTINGS
\setstretch{1}
\allowdisplaybreaks
\setlength{\parindent}{0pt} %Sets paragraph indent to 0pt
\hypersetup{
	colorlinks=true,
	linkcolor=black,
	filecolor=black,      
	urlcolor=black,
}

\begin{document}

\thispagestyle{firstpage}

\begin{center}
	\vspace*{-15pt}
	{\huge\textsc{\textbf{Sarah Robinson}}} \\[2pt]
\end{center}

\vspace{-6pt}

\noindent\rule{\textwidth}{1pt}

\vspace{3pt}

\begin{tabularx}{\textwidth}{@{}Y Z@{}}
	
	%%%%%%%%%%%%%%%%%%%%%%%%%%%%%%%%%%%%%%%%%%%%%%%%%%%%%%%%%%%%%%%%%%%%%%%%%%%%%%%%%%%%%%%%%%%%%%%
	\textsc{Contact \newline Information} & 
	\begin{minipage}[t]{0.35\textwidth}
		2015 North Hall \newline
		Department of Economics \newline
		University of California \newline
		Santa Barbara, CA 93106
	\end{minipage}\begin{minipage}[t]{0.4\textwidth}
	\makebox[45pt]{\textit{Phone:} \hfill} (805) 316-1377 \newline
	\makebox[45pt]{\textit{E-mail:} \hfill} sarahrobinson@ucsb.edu \newline
	\makebox[45pt]{\textit{Website:} \hfill} \href{https://www.s-robinson.com}{\color{blue}{www.s-robinson.com}}
	\end{minipage}
	\newline  \\ \addlinespace[15pt] 
	
	%%%%%%%%%%%%%%%%%%%%%%%%%%%%%%%%%%%%%%%%%%%%%%%%%%%%%%%%%%%%%%%%%%%%%%%%%%%%%%%%%%%%%%%%%%%%%%%
	\textsc{Citizenship} & 
	U.S.
	\\ \addlinespace[20pt] 
	
	%%%%%%%%%%%%%%%%%%%%%%%%%%%%%%%%%%%%%%%%%%%%%%%%%%%%%%%%%%%%%%%%%%%%%%%%%%%%%%%%%%%%%%%%%%%%%%%
	\textsc{Research Fields} & 
	Health Economics, Public Economics, Applied Microeconomics
	 \\ \addlinespace[20pt] 
	
	
	%%%%%%%%%%%%%%%%%%%%%%%%%%%%%%%%%%%%%%%%%%%%%%%%%%%%%%%%%%%%%%%%%%%%%%%%%%%%%%%%%%%%%%%%%%%%%%%
	\textsc{Education} &
	\textbf{University of California, Santa Barbara}
	\vspace{3pt} \newline
	\hspace*{15pt} Ph.D., Economics \hfill Expected June 2023%
	\vspace{3pt} \newline
	\hspace*{15pt} \textit{Dissertation Committee: Heather Royer (advisor), \newline \hspace*{15pt} Youssef Benzarti, H.E. (Ted) Frech III, Alisa Tazhitdinova} 
	\vspace{3pt} \newline
	\hspace*{15pt} M.A., Economics \hfill 2018%
	\vspace{10pt} \newline
	\textbf{Claremont McKenna College,} Claremont, CA 
	\vspace{3pt} \newline
	\hspace*{15pt} B.A., Philosophy, Politics \& Economics, \textit{magna cum laude} \hfill 2013%
    \\  \addlinespace[20pt] 

	%%%%%%%%%%%%%%%%%%%%%%%%%%%%%%%%%%%%%%%%%%%%%%%%%%%%%%%%%%%%%%%%%%%%%%%%%%%%%%%%%%%%%%%%%%%%%%%
	\textsc{Working \newline Papers} \vspace{10pt} \newline \textsc{(Abstracts \newline at End)} & 
	 \textbf{(Job Market Paper)} ``Do Firms Avoid Health Insurance Mandates? \newline Evidence from the Self-Funding of Employer Plans"
	\vspace{10pt} \newline
	``Corporate Political Spending and State Tax Policy: Evidence from Citizens United" \newline with Cailin Slattery and Alisa Tazhitdinova \textit{(NBER Working Paper 30352)} \newline \textbf{(Revise \& Resubmit, Journal of Public Economics)}
	\vspace{10pt} \newline
	``What Drives Tax Policy? Political, Institutional and Economic Determinants \newline of State Tax Policy in the Past 70 Years" with Alisa Tazhitdinova \textit{(under review)}
	\vspace{10pt} \newline
	``Geographic Variation in C-Sections in the United States: Trends, Correlates \newline and Other Interesting Facts" with Heather Royer and David Silver \textit{(under review)}
     \\ \addlinespace[10pt] 
    
    %%%%%%%%%%%%%%%%%%%%%%%%%%%%%%%%%%%%%%%%%%%%%%%%%%%%%%%%%%%%%%%%%%%%%%%%%%%%%%%%%%%%%%%%%%%%%%%
    \textsc{Work in \newline Progress}  & 
    ``Employer Choice of Health Insurance Plans and Premium Sharing" \newline
    \textit{(FSRDC project approved by U.S. Census Bureau and IRS)} 
     \\ \addlinespace[20pt]

	 %%%%%%%%%%%%%%%%%%%%%%%%%%%%%%%%%%%%%%%%%%%%%%%%%%%%%%%%%%%%%%%%%%%%%%%%%%%%%%%%%%%%%%%%%%%%%%%
	\textsc{Affiliations}  & 
	U.S. Census Bureau, Special Sworn Status Researcher \hfill 2022 -- present%
	\\ \addlinespace[20pt]

	%%%%%%%%%%%%%%%%%%%%%%%%%%%%%%%%%%%%%%%%%%%%%%%%%%%%%%%%%%%%%%%%%%%%%%%%%%%%%%%%%%%%%%%%%%%%%%%
	\textsc{Fellowships \newline \& Awards} & 
	\textbf{National Bureau of Economic Research,} Cambridge, MA  \hfill 2021 -- 2022%
	\newline Pre-Doctoral Fellow, Aging \& Health Research
	\vspace{12pt} \newline
	\uline{University of California, Santa Barbara}
	\vspace{3pt} \newline
	Job Market Fellowship \hfill 2022%
	\vspace{3pt} \newline
	Outstanding Teaching Assistant Award \hfill 2019, 2021%
	\vspace{3pt} \newline
	Research Quarter Fellowship \hfill 2019%
	\vspace{3pt} \newline
	Gretler Fellowship \hfill 2019%
	\vspace{3pt} \newline 
	Mortimer Andron Fellowship \hfill 2018 -- 2022%
	\\ \newpage 
	&
	%\vspace{3pt} \newline 
	Distinction in Ph.D. Preliminary Exam, Microeconomics \hfill 2018%
	\vspace{3pt} \newline
	Regents Fellowship \hfill 2017 -- 2018%
	\vspace{12pt} \newline
	\uline{Claremont McKenna College}
	\vspace{3pt} \newline
	Phi Beta Kappa \hfill 2013%
	\vspace{3pt} \newline
	Edward J. Sexton PPE Fellow \hfill 2011 -- 2013%
	\vspace{3pt} \newline
	Dean's List \hfill 2010 -- 2013%
	 \\ \addlinespace[20pt] 
	
	%%%%%%%%%%%%%%%%%%%%%%%%%%%%%%%%%%%%%%%%%%%%%%%%%%%%%%%%%%%%%%%%%%%%%%%%%%%%%%%%%%%%%%%%%%%%%%%
	\textsc{Research \vspace{3pt} \newline Assistance}  & 
	Heather Royer \& David Silver \hfill 2020 -- 2021%
	\vspace{3pt} \newline 
	Youssef Benzarti \& Alisa Tazhitdinova \hfill 2019%
	 \\ \addlinespace[20pt] 
	
	%%%%%%%%%%%%%%%%%%%%%%%%%%%%%%%%%%%%%%%%%%%%%%%%%%%%%%%%%%%%%%%%%%%%%%%%%%%%%%%%%%%%%%%%%%%%%%%
	\textsc{Teaching \vspace{3pt} \newline Experience}  & 
	\uline{University of California, Santa Barbara}
	\vspace{3pt} \newline
	\textit{* Average rating 1.3 (1 = highest, 5 = lowest)}
	\vspace{3pt} \newline
	Instructor, Math Camp for Economics Ph.D. Students \hfill Su. 2020%
	\vspace{3pt} \newline
	Head Teaching Assistant, Economics 10A Microeconomic Theory \hfill Sp. 2020%
	\vspace{3pt} \newline
	Teaching Assistant, Economics 10A Microeconomic Theory \textit{(6 quarters)} \hfill 2018 -- 2022%
	\vspace{3pt} \newline
	Teaching Assistant, Economics 134A Financial Management  \hfill W. \& Sp. 2019%
	\vspace{15pt} \newline
	\uline{Claremont McKenna College}
	\vspace{3pt} \newline
	Class Tutor, Economics 120 Statistics \hfill F. 2012%
	 \\ \addlinespace[20pt] 
	
	%%%%%%%%%%%%%%%%%%%%%%%%%%%%%%%%%%%%%%%%%%%%%%%%%%%%%%%%%%%%%%%%%%%%%%%%%%%%%%%%%%%%%%%%%%%%%%%
	\textsc{Presentations} & 
	All-California Labor Economics Conference \textit{(poster session)}  \hfill  2022%
	\vspace{3pt} \newline
	Online Public Finance Seminar Graduate Student Workshop \hfill  2022%  
	\vspace{3pt} \newline
	UCSB Applied Microeconomics Lunch  \hfill  2021, 2022%
	\vspace{3pt} \newline
	NBER Spring Public Economics Meeting \textit{(co-author presenting)}  \hfill  2022%  
	\vspace{3pt} \newline
	AEA Annual Meeting \textit{(poster session)}  \hfill  2022% 
	\vspace{3pt} \newline
	MannheimTaxation  \hfill  2021% 
	 \\ \addlinespace[20pt] 
	
	%%%%%%%%%%%%%%%%%%%%%%%%%%%%%%%%%%%%%%%%%%%%%%%%%%%%%%%%%%%%%%%%%%%%%%%%%%%%%%%%%%%%%%%%%%%%%%%
	\textsc{Workshops} & 
	NBER Health Economics Research Bootcamp \hfill 2019%
	\vspace{6pt} \newline
	$\nth{7}$ Lindau Meeting on Economic Sciences \newline \textit{(successfully nominated by University of California but unable to attend)} 
	\\ \addlinespace[20pt] 
	
	%%%%%%%%%%%%%%%%%%%%%%%%%%%%%%%%%%%%%%%%%%%%%%%%%%%%%%%%%%%%%%%%%%%%%%%%%%%%%%%%%%%%%%%%%%%%%%%
	\textsc{Refereeing}  & 
	American Economic Journal: Economic Policy
	 \\ \addlinespace[20pt] 
	
	%%%%%%%%%%%%%%%%%%%%%%%%%%%%%%%%%%%%%%%%%%%%%%%%%%%%%%%%%%%%%%%%%%%%%%%%%%%%%%%%%%%%%%%%%%%%%%%
	\textsc{Service} & 
	Mentor for First-Year Economics Ph.D. Students, UCSB \hfill 2018 -- 2021%
	\vspace{3pt} \newline
	Professional Development Committee, UCSB \textit{(co-founder)}  \hfill 2019 -- 2020%
	\vspace{3pt} \newline
	Health Reading Group, UCSB \textit{(organizer)} \hfill 2019 -- 2020% 
	\\ \addlinespace[20pt] 
	

	%%%%%%%%%%%%%%%%%%%%%%%%%%%%%%%%%%%%%%%%%%%%%%%%%%%%%%%%%%%%%%%%%%%%%%%%%%%%%%%%%%%%%%%%%%%%%%%
	\textsc{Other Work \newline Experience}  & 
	\textbf{Deloitte Consulting LLP,} San Francisco, CA
	\vspace{0pt} \newline
	\textbf{Strategy \& Operations} 
	\vspace{3pt} \newline
	\textit{Consultant} \hfill 2015 -- 2016%
	\vspace{0pt} \newline
	\textit{Business Analyst} \hfill 2013 -- 2015%
	\\ \newpage
	%\\ \addlinespace[20pt] 
	
	%%%%%%%%%%%%%%%%%%%%%%%%%%%%%%%%%%%%%%%%%%%%%%%%%%%%%%%%%%%%%%%%%%%%%%%%%%%%%%%%%%%%%%%%%%%%%%%
	\textsc{Computer  \vspace*{3pt} \newline Skills}  & 
	Stata, Python, Excel, PowerPoint, \LaTeX \ \textit{(advanced)}
	\vspace{3pt} \newline
	R, MATLAB \textit{(basic)}
	\\ \addlinespace[20pt] 
	
		
	%%%%%%%%%%%%%%%%%%%%%%%%%%%%%%%%%%%%%%%%%%%%%%%%%%%%%%%%%%%%%%%%%%%%%%%%%%%%%%%%%%%%%%%%%%%%%%%
	\textsc{References}  & 
	\begin{minipage}[t]{0.41\textwidth}
		\textbf{Heather Royer} \textit{(advisor)} \newline
		Professor \newline
		Department of Economics \newline
		University of California, Santa Barbara \newline
		heather.royer@ucsb.edu 
		%%%
		\vspace{20pt} \newline
		\textbf{H.E. (Ted) Frech, III} \newline
		Professor \newline
		Department of Economics \newline
		University of California, Santa Barbara \newline
		frech@ucsb.edu
	\end{minipage}\begin{minipage}[t]{0.5\textwidth}
		\textbf{Youssef Benzarti}  \newline
		Associate Professor \newline
		Department of Economics \newline
		University of California, Santa Barbara \newline
		benzarti@ucsb.edu
		%%%
		\vspace{20pt} \newline
		\textbf{Alisa Tazhitdinova} \newline
		Assistant Professor \newline
		Department of Economics \newline
		University of California, Santa Barbara \newline
		tazhitda@ucsb.edu
	\end{minipage}
	\newline  \\ \addlinespace[20pt] 
	
	%%%%%%%%%%%%%%%%%%%%%%%%%%%%%%%%%%%%%%%%%%%%%%%%%%%%%%%%%%%%%%%%%%%%%%%%%%%%%%%%%%%%%%%%%%%%%%%
	\textsc{Additional \newline Contacts}  & 
	\begin{minipage}[t]{0.41\textwidth}
		\textbf{Placement Director} \newline
		Professor Erik Eyster \newline
		Department of Economics \newline
		University of California, Santa Barbara \newline
		erikeyster@ucsb.edu
	\end{minipage}\begin{minipage}[t]{0.5\textwidth}
		\textbf{Placement Administrator} \newline
		Mark Patterson \newline
		Department of Economics \newline
		University of California, Santa Barbara \newline
		grad@econ.ucsb.edu
	\end{minipage}
	\newline  \\ \addlinespace[20pt] 
	
	%%%%%%%%%%%%%%%%%%%%%%%%%%%%%%%%%%%%%%%%%%%%%%%%%%%%%%%%%%%%%%%%%%%%%%%%%%%%%%%%%%%%%%%%%%%%%%%
	\textsc{Working \newline Papers -- \newline Abstracts}  & 
	\textbf{(Job Market Paper) ``Do Firms Avoid Health Insurance Mandates? \newline Evidence from the Self-Funding of Employer Plans"} \newline
	\vspace{-8pt} \newline
Fifty percent of the U.S. population gets health insurance through an employer, and roughly half of employers only offer one health plan. Therefore, the choices made by firms about what plan(s) to offer are critical to understanding the health insurance available to workers. This paper focuses on one dimension of the firm's decision: whether to self-fund plans (meaning the firm bears the financial risk of claims itself). I study whether firms use self-funding to avoid complying with mandates to cover specific procedures or providers. Using administrative data on the health plans offered by firms and a difference-in-differences design, I find that new mandates increase rates of self-funding among smaller firms (100-249 employees) by 3.2 percentage points, an increase of 14.5\%. The mandates do not appear to affect larger firms (250+ employees), who are more likely to already be self-funded in the pre-period. These results imply that new mandates can lead to long-lasting reductions in the proportion of firms that are bound by any state health insurance regulations, including all previously mandated benefits as well as premium taxes.  \href{https://www.s-robinson.com/Robinson_JMP.pdf}{\color{blue}{[Link to paper]}}
	\vspace{20pt} \newline
		\textbf{``Corporate Political Spending and State Tax Policy: Evidence from \newline  Citizens United" with Cailin Slattery and Alisa Tazhitdinova} \newline \textit{(Revise \& Resubmit, Journal of Public Economics)} \newline
	\vspace{-8pt} \newline
	To what extent is U.S. state tax policy affected by corporate political contributions? The 2010 Supreme Court \textit{Citizens United v. Federal Election Commission} ruling provides an exogenous shock to corporate campaign spending, allowing corporations to spend on elections in 23 states which previously had spending bans. Ten years after the ruling and for a wide range of outcomes, we are not able to identify economically or statistically significant effects of corporate independent expenditures on state tax policy, including tax rates, discretionary tax breaks, and tax revenues. \href{https://www.s-robinson.com/research/Slattery_Tazhitdinova_Robinson_Citizens_United.pdf}{\color{blue}{[Link to paper]}}
	\\
	\newpage
	
	&
	\textbf{``What Drives Tax Policy? Political, Institutional and Economic Determinants  \newline of State Tax Policy in the Past 70 Years" with Alisa Tazhitdinova} \textit{(under review)} 
	\vspace{-8pt} \newline 
	We study U.S. state tax rules over the past 70 years to shed light on the determinants of U.S. state tax policy, generating three key results. First, we show that long-term tax trends are not consistent with Tiebout sorting and race-to-the-bottom competition models. Second, we document evidence of increasing polarization of tax rates between Democratic and Republican states in the 1970s and from 2000 onward. Third, we use machine learning techniques to show that the timing and magnitude of tax changes are not driven by federal changes, economic needs, state politics, institutional rules, neighbor competition, or demographics. Altogether, these factors explain less than 20\% of observed tax variation.  \vspace{2pt} \newline  \href{https://www.s-robinson.com/research/Robinson_Tazhitdinova_Tax_Policy_Determinants.pdf}{\color{blue}{[Link to paper]}}
	\vspace{20pt} \newline
	\textbf{``Geographic Variation in C-Sections in the United States: Trends, Correlates \newline and Other Interesting Facts" with Heather Royer and David Silver} \textit{(under review)} 
	\vspace{-8pt} \newline
	We use U.S. natality data from 1989 to 2017 to investigate county-level geographic disparities in the use of C‑section among first-birth singleton mothers. We document the existence and persistence of geographic variation in C‑section across low and high- C‑section risk mothers, the degree to which this variation correlates with Medicare spending, and the sensitivity of C‑section use and infant and maternal health outcomes to C‑section risk across counties. Our key finding is that counties with high C‑section rates perform more C‑sections across the entirety of the risk distribution yet have nearly equivalent or better outcomes than counties with less intensive C‑section rates.  \href{https://www.s-robinson.com/research/Robinson_Royer_Silver_CSection.pdf}{\color{blue}{[Link to paper]}}
	\vspace{0pt}
	\\ \addlinespace[20pt] 
	


\end{tabularx}

\end{document}