%Document Class
	\documentclass{article}
	
%Packages
	\usepackage{amsmath}	%Enables math stuff
	\usepackage{amssymb}	%Enables math stuff
	\usepackage[margin=0.75 in]{geometry}	% Affects document settings
	
%Other Document Settings
	\setlength\parindent{0pt}	%Sets paragraph indent
	\renewcommand{\arraystretch}{1.25}	%Sets vertical spacing between lines in a table

%Document
\begin{document}


Name\\
Course\\
Date \\[10pt]

\begin{enumerate}
%%%%%%%%%%%%%
%PROBLEM ONE%
%%%%%%%%%%%%%
\item\textbf{This is where the text goes for question one. What is 2+2+2?} \\[5pt]
	I'm going to do some simple addition as an example.
	\begin{align*} %Opens an align environment. Document won't compile if there are any empty lines within the align environment!
	2 + 2 + 2 & = (2 + 2) + 2 &\text{(by associativity)} \\ %Within a math environment (such as align), math mode is the default so you need to use \text{} for words
	& = 4 + 2 & \text{(adding inside the parentheses)} \\
	& \boxed{ = 6} & \text{(by simple addition)} %Puts a box around the answer
	\end{align*}
	This completes the first ``problem." 
	
	\vspace{10pt} %Creates vertical space before the next problem

%%%%%%%%%%%%%
%PROBLEM TWO%
%%%%%%%%%%%%%
\item\textbf{This is where the text goes for question two.}
	\begin{enumerate}
		\item\textbf{Question two, subpart (a).}
		\item\textbf{Question two, subpart (b).}
	\end{enumerate}
	
	\vspace{20pt}
	
%%%%%%%%%%%%%
%TRUTH TABLE%
%%%%%%%%%%%%%
\item\textbf{Example Truth Table}
	\begin{center}
			\begin{tabular}{c c c c} %Opens a tabular environment, for a table with 4 columns that are center-spaced
				$P$		&$Q$		&$\neg P$	&$\neg Q$	\\ %If you aren't in a math environment, use $math stuff$ to get to math mode
				\hline
				T		&T		&F			&F			\\
				T		&F		&F			&T			\\
				F		&T		&T			&F			\\
				F		&F		&T			&T					
		\end{tabular}
	\end{center}

	\vspace{10pt}


%%%%%%%%%%%%%
%Proof%
%%%%%%%%%%%%%

\item\textbf{Example Proof}

	\underline{To Show:} $R$ \\
	\underline{Proof:} 
	\begin{align*}
		& \text{Let P} & \text{(by hypothesis)} \\
		& \implies Q & \text{(Theorem 1)} \\
		& \implies R & \text{(Theorem 2)} \\
		&& \blacksquare \\
	\end{align*}

	\vspace{10pt}

%%%%%%%%%%%%%
%Equation Mode%
%%%%%%%%%%%%%

\item\textbf{Example Equation Mode}

	\[ \mathbb{R}^+ = \{ x \mid ( x \in \mathbb{R} ) \wedge (x \geq 0)  \} \]
	%To create an equation that's centered on its own line, use \[ math stuff \] to get to math mode
	\[ y = \frac{1}{2} \hspace{20mm} 55 = \sum_{i = 1}^{10} i \hspace{10mm} \hspace{10mm} \hat{\beta} = (\mathbf{X'X})^{-1} \mathbf{X'Y} \]  %Fractions, summations, and matrix notation
	
	\vspace{10pt}


%%%%%%%%%%%%%
%Matrices%
%%%%%%%%%%%%%

\item\textbf{Example Matrices}


	\[
	A=\begin{bmatrix} 1 & 2 \\ 3 & 4\end{bmatrix} 	
	\hspace{10mm}  
	B=\begin{bmatrix} b_{11} & b_{12} \\ b_{21} & b_{22} \\ b_{31} &  b_{32} \end{bmatrix} 	
	\]


\end{enumerate}

\end{document}